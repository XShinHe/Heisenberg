
% Default to the notebook output style

    


% Inherit from the specified cell style.




    
\documentclass[11pt]{article}

    
    
    \usepackage[T1]{fontenc}
    % Nicer default font (+ math font) than Computer Modern for most use cases
    \usepackage{mathpazo}

    % Basic figure setup, for now with no caption control since it's done
    % automatically by Pandoc (which extracts ![](path) syntax from Markdown).
    \usepackage{graphicx}
    % We will generate all images so they have a width \maxwidth. This means
    % that they will get their normal width if they fit onto the page, but
    % are scaled down if they would overflow the margins.
    \makeatletter
    \def\maxwidth{\ifdim\Gin@nat@width>\linewidth\linewidth
    \else\Gin@nat@width\fi}
    \makeatother
    \let\Oldincludegraphics\includegraphics
    % Set max figure width to be 80% of text width, for now hardcoded.
    \renewcommand{\includegraphics}[1]{\Oldincludegraphics[width=.8\maxwidth]{#1}}
    % Ensure that by default, figures have no caption (until we provide a
    % proper Figure object with a Caption API and a way to capture that
    % in the conversion process - todo).
    \usepackage{caption}
    \DeclareCaptionLabelFormat{nolabel}{}
    \captionsetup{labelformat=nolabel}

    \usepackage{adjustbox} % Used to constrain images to a maximum size 
    \usepackage{xcolor} % Allow colors to be defined
    \usepackage{enumerate} % Needed for markdown enumerations to work
    \usepackage{geometry} % Used to adjust the document margins
    \usepackage{amsmath} % Equations
    \usepackage{amssymb} % Equations
    \usepackage{textcomp} % defines textquotesingle
    % Hack from http://tex.stackexchange.com/a/47451/13684:
    \AtBeginDocument{%
        \def\PYZsq{\textquotesingle}% Upright quotes in Pygmentized code
    }
    \usepackage{upquote} % Upright quotes for verbatim code
    \usepackage{eurosym} % defines \euro
    \usepackage[mathletters]{ucs} % Extended unicode (utf-8) support
    \usepackage[utf8x]{inputenc} % Allow utf-8 characters in the tex document
    \usepackage{fancyvrb} % verbatim replacement that allows latex
    \usepackage{grffile} % extends the file name processing of package graphics 
                         % to support a larger range 
    % The hyperref package gives us a pdf with properly built
    % internal navigation ('pdf bookmarks' for the table of contents,
    % internal cross-reference links, web links for URLs, etc.)
    \usepackage{hyperref}
    \usepackage{longtable} % longtable support required by pandoc >1.10
    \usepackage{booktabs}  % table support for pandoc > 1.12.2
    \usepackage[inline]{enumitem} % IRkernel/repr support (it uses the enumerate* environment)
    \usepackage[normalem]{ulem} % ulem is needed to support strikethroughs (\sout)
                                % normalem makes italics be italics, not underlines
    

    
    
    % Colors for the hyperref package
    \definecolor{urlcolor}{rgb}{0,.145,.698}
    \definecolor{linkcolor}{rgb}{.71,0.21,0.01}
    \definecolor{citecolor}{rgb}{.12,.54,.11}

    % ANSI colors
    \definecolor{ansi-black}{HTML}{3E424D}
    \definecolor{ansi-black-intense}{HTML}{282C36}
    \definecolor{ansi-red}{HTML}{E75C58}
    \definecolor{ansi-red-intense}{HTML}{B22B31}
    \definecolor{ansi-green}{HTML}{00A250}
    \definecolor{ansi-green-intense}{HTML}{007427}
    \definecolor{ansi-yellow}{HTML}{DDB62B}
    \definecolor{ansi-yellow-intense}{HTML}{B27D12}
    \definecolor{ansi-blue}{HTML}{208FFB}
    \definecolor{ansi-blue-intense}{HTML}{0065CA}
    \definecolor{ansi-magenta}{HTML}{D160C4}
    \definecolor{ansi-magenta-intense}{HTML}{A03196}
    \definecolor{ansi-cyan}{HTML}{60C6C8}
    \definecolor{ansi-cyan-intense}{HTML}{258F8F}
    \definecolor{ansi-white}{HTML}{C5C1B4}
    \definecolor{ansi-white-intense}{HTML}{A1A6B2}

    % commands and environments needed by pandoc snippets
    % extracted from the output of `pandoc -s`
    \providecommand{\tightlist}{%
      \setlength{\itemsep}{0pt}\setlength{\parskip}{0pt}}
    \DefineVerbatimEnvironment{Highlighting}{Verbatim}{commandchars=\\\{\}}
    % Add ',fontsize=\small' for more characters per line
    \newenvironment{Shaded}{}{}
    \newcommand{\KeywordTok}[1]{\textcolor[rgb]{0.00,0.44,0.13}{\textbf{{#1}}}}
    \newcommand{\DataTypeTok}[1]{\textcolor[rgb]{0.56,0.13,0.00}{{#1}}}
    \newcommand{\DecValTok}[1]{\textcolor[rgb]{0.25,0.63,0.44}{{#1}}}
    \newcommand{\BaseNTok}[1]{\textcolor[rgb]{0.25,0.63,0.44}{{#1}}}
    \newcommand{\FloatTok}[1]{\textcolor[rgb]{0.25,0.63,0.44}{{#1}}}
    \newcommand{\CharTok}[1]{\textcolor[rgb]{0.25,0.44,0.63}{{#1}}}
    \newcommand{\StringTok}[1]{\textcolor[rgb]{0.25,0.44,0.63}{{#1}}}
    \newcommand{\CommentTok}[1]{\textcolor[rgb]{0.38,0.63,0.69}{\textit{{#1}}}}
    \newcommand{\OtherTok}[1]{\textcolor[rgb]{0.00,0.44,0.13}{{#1}}}
    \newcommand{\AlertTok}[1]{\textcolor[rgb]{1.00,0.00,0.00}{\textbf{{#1}}}}
    \newcommand{\FunctionTok}[1]{\textcolor[rgb]{0.02,0.16,0.49}{{#1}}}
    \newcommand{\RegionMarkerTok}[1]{{#1}}
    \newcommand{\ErrorTok}[1]{\textcolor[rgb]{1.00,0.00,0.00}{\textbf{{#1}}}}
    \newcommand{\NormalTok}[1]{{#1}}
    
    % Additional commands for more recent versions of Pandoc
    \newcommand{\ConstantTok}[1]{\textcolor[rgb]{0.53,0.00,0.00}{{#1}}}
    \newcommand{\SpecialCharTok}[1]{\textcolor[rgb]{0.25,0.44,0.63}{{#1}}}
    \newcommand{\VerbatimStringTok}[1]{\textcolor[rgb]{0.25,0.44,0.63}{{#1}}}
    \newcommand{\SpecialStringTok}[1]{\textcolor[rgb]{0.73,0.40,0.53}{{#1}}}
    \newcommand{\ImportTok}[1]{{#1}}
    \newcommand{\DocumentationTok}[1]{\textcolor[rgb]{0.73,0.13,0.13}{\textit{{#1}}}}
    \newcommand{\AnnotationTok}[1]{\textcolor[rgb]{0.38,0.63,0.69}{\textbf{\textit{{#1}}}}}
    \newcommand{\CommentVarTok}[1]{\textcolor[rgb]{0.38,0.63,0.69}{\textbf{\textit{{#1}}}}}
    \newcommand{\VariableTok}[1]{\textcolor[rgb]{0.10,0.09,0.49}{{#1}}}
    \newcommand{\ControlFlowTok}[1]{\textcolor[rgb]{0.00,0.44,0.13}{\textbf{{#1}}}}
    \newcommand{\OperatorTok}[1]{\textcolor[rgb]{0.40,0.40,0.40}{{#1}}}
    \newcommand{\BuiltInTok}[1]{{#1}}
    \newcommand{\ExtensionTok}[1]{{#1}}
    \newcommand{\PreprocessorTok}[1]{\textcolor[rgb]{0.74,0.48,0.00}{{#1}}}
    \newcommand{\AttributeTok}[1]{\textcolor[rgb]{0.49,0.56,0.16}{{#1}}}
    \newcommand{\InformationTok}[1]{\textcolor[rgb]{0.38,0.63,0.69}{\textbf{\textit{{#1}}}}}
    \newcommand{\WarningTok}[1]{\textcolor[rgb]{0.38,0.63,0.69}{\textbf{\textit{{#1}}}}}
    
    
    % Define a nice break command that doesn't care if a line doesn't already
    % exist.
    \def\br{\hspace*{\fill} \\* }
    % Math Jax compatability definitions
    \def\gt{>}
    \def\lt{<}
    % Document parameters
    \title{5th Homework Tutorial of Quantum Chemistry Programming - ver.01 }
    \author{XShinHe}
    \date{}
    
    

    % Pygments definitions
    
\makeatletter
\def\PY@reset{\let\PY@it=\relax \let\PY@bf=\relax%
    \let\PY@ul=\relax \let\PY@tc=\relax%
    \let\PY@bc=\relax \let\PY@ff=\relax}
\def\PY@tok#1{\csname PY@tok@#1\endcsname}
\def\PY@toks#1+{\ifx\relax#1\empty\else%
    \PY@tok{#1}\expandafter\PY@toks\fi}
\def\PY@do#1{\PY@bc{\PY@tc{\PY@ul{%
    \PY@it{\PY@bf{\PY@ff{#1}}}}}}}
\def\PY#1#2{\PY@reset\PY@toks#1+\relax+\PY@do{#2}}

\expandafter\def\csname PY@tok@w\endcsname{\def\PY@tc##1{\textcolor[rgb]{0.73,0.73,0.73}{##1}}}
\expandafter\def\csname PY@tok@c\endcsname{\let\PY@it=\textit\def\PY@tc##1{\textcolor[rgb]{0.25,0.50,0.50}{##1}}}
\expandafter\def\csname PY@tok@cp\endcsname{\def\PY@tc##1{\textcolor[rgb]{0.74,0.48,0.00}{##1}}}
\expandafter\def\csname PY@tok@k\endcsname{\let\PY@bf=\textbf\def\PY@tc##1{\textcolor[rgb]{0.00,0.50,0.00}{##1}}}
\expandafter\def\csname PY@tok@kp\endcsname{\def\PY@tc##1{\textcolor[rgb]{0.00,0.50,0.00}{##1}}}
\expandafter\def\csname PY@tok@kt\endcsname{\def\PY@tc##1{\textcolor[rgb]{0.69,0.00,0.25}{##1}}}
\expandafter\def\csname PY@tok@o\endcsname{\def\PY@tc##1{\textcolor[rgb]{0.40,0.40,0.40}{##1}}}
\expandafter\def\csname PY@tok@ow\endcsname{\let\PY@bf=\textbf\def\PY@tc##1{\textcolor[rgb]{0.67,0.13,1.00}{##1}}}
\expandafter\def\csname PY@tok@nb\endcsname{\def\PY@tc##1{\textcolor[rgb]{0.00,0.50,0.00}{##1}}}
\expandafter\def\csname PY@tok@nf\endcsname{\def\PY@tc##1{\textcolor[rgb]{0.00,0.00,1.00}{##1}}}
\expandafter\def\csname PY@tok@nc\endcsname{\let\PY@bf=\textbf\def\PY@tc##1{\textcolor[rgb]{0.00,0.00,1.00}{##1}}}
\expandafter\def\csname PY@tok@nn\endcsname{\let\PY@bf=\textbf\def\PY@tc##1{\textcolor[rgb]{0.00,0.00,1.00}{##1}}}
\expandafter\def\csname PY@tok@ne\endcsname{\let\PY@bf=\textbf\def\PY@tc##1{\textcolor[rgb]{0.82,0.25,0.23}{##1}}}
\expandafter\def\csname PY@tok@nv\endcsname{\def\PY@tc##1{\textcolor[rgb]{0.10,0.09,0.49}{##1}}}
\expandafter\def\csname PY@tok@no\endcsname{\def\PY@tc##1{\textcolor[rgb]{0.53,0.00,0.00}{##1}}}
\expandafter\def\csname PY@tok@nl\endcsname{\def\PY@tc##1{\textcolor[rgb]{0.63,0.63,0.00}{##1}}}
\expandafter\def\csname PY@tok@ni\endcsname{\let\PY@bf=\textbf\def\PY@tc##1{\textcolor[rgb]{0.60,0.60,0.60}{##1}}}
\expandafter\def\csname PY@tok@na\endcsname{\def\PY@tc##1{\textcolor[rgb]{0.49,0.56,0.16}{##1}}}
\expandafter\def\csname PY@tok@nt\endcsname{\let\PY@bf=\textbf\def\PY@tc##1{\textcolor[rgb]{0.00,0.50,0.00}{##1}}}
\expandafter\def\csname PY@tok@nd\endcsname{\def\PY@tc##1{\textcolor[rgb]{0.67,0.13,1.00}{##1}}}
\expandafter\def\csname PY@tok@s\endcsname{\def\PY@tc##1{\textcolor[rgb]{0.73,0.13,0.13}{##1}}}
\expandafter\def\csname PY@tok@sd\endcsname{\let\PY@it=\textit\def\PY@tc##1{\textcolor[rgb]{0.73,0.13,0.13}{##1}}}
\expandafter\def\csname PY@tok@si\endcsname{\let\PY@bf=\textbf\def\PY@tc##1{\textcolor[rgb]{0.73,0.40,0.53}{##1}}}
\expandafter\def\csname PY@tok@se\endcsname{\let\PY@bf=\textbf\def\PY@tc##1{\textcolor[rgb]{0.73,0.40,0.13}{##1}}}
\expandafter\def\csname PY@tok@sr\endcsname{\def\PY@tc##1{\textcolor[rgb]{0.73,0.40,0.53}{##1}}}
\expandafter\def\csname PY@tok@ss\endcsname{\def\PY@tc##1{\textcolor[rgb]{0.10,0.09,0.49}{##1}}}
\expandafter\def\csname PY@tok@sx\endcsname{\def\PY@tc##1{\textcolor[rgb]{0.00,0.50,0.00}{##1}}}
\expandafter\def\csname PY@tok@m\endcsname{\def\PY@tc##1{\textcolor[rgb]{0.40,0.40,0.40}{##1}}}
\expandafter\def\csname PY@tok@gh\endcsname{\let\PY@bf=\textbf\def\PY@tc##1{\textcolor[rgb]{0.00,0.00,0.50}{##1}}}
\expandafter\def\csname PY@tok@gu\endcsname{\let\PY@bf=\textbf\def\PY@tc##1{\textcolor[rgb]{0.50,0.00,0.50}{##1}}}
\expandafter\def\csname PY@tok@gd\endcsname{\def\PY@tc##1{\textcolor[rgb]{0.63,0.00,0.00}{##1}}}
\expandafter\def\csname PY@tok@gi\endcsname{\def\PY@tc##1{\textcolor[rgb]{0.00,0.63,0.00}{##1}}}
\expandafter\def\csname PY@tok@gr\endcsname{\def\PY@tc##1{\textcolor[rgb]{1.00,0.00,0.00}{##1}}}
\expandafter\def\csname PY@tok@ge\endcsname{\let\PY@it=\textit}
\expandafter\def\csname PY@tok@gs\endcsname{\let\PY@bf=\textbf}
\expandafter\def\csname PY@tok@gp\endcsname{\let\PY@bf=\textbf\def\PY@tc##1{\textcolor[rgb]{0.00,0.00,0.50}{##1}}}
\expandafter\def\csname PY@tok@go\endcsname{\def\PY@tc##1{\textcolor[rgb]{0.53,0.53,0.53}{##1}}}
\expandafter\def\csname PY@tok@gt\endcsname{\def\PY@tc##1{\textcolor[rgb]{0.00,0.27,0.87}{##1}}}
\expandafter\def\csname PY@tok@err\endcsname{\def\PY@bc##1{\setlength{\fboxsep}{0pt}\fcolorbox[rgb]{1.00,0.00,0.00}{1,1,1}{\strut ##1}}}
\expandafter\def\csname PY@tok@kc\endcsname{\let\PY@bf=\textbf\def\PY@tc##1{\textcolor[rgb]{0.00,0.50,0.00}{##1}}}
\expandafter\def\csname PY@tok@kd\endcsname{\let\PY@bf=\textbf\def\PY@tc##1{\textcolor[rgb]{0.00,0.50,0.00}{##1}}}
\expandafter\def\csname PY@tok@kn\endcsname{\let\PY@bf=\textbf\def\PY@tc##1{\textcolor[rgb]{0.00,0.50,0.00}{##1}}}
\expandafter\def\csname PY@tok@kr\endcsname{\let\PY@bf=\textbf\def\PY@tc##1{\textcolor[rgb]{0.00,0.50,0.00}{##1}}}
\expandafter\def\csname PY@tok@bp\endcsname{\def\PY@tc##1{\textcolor[rgb]{0.00,0.50,0.00}{##1}}}
\expandafter\def\csname PY@tok@fm\endcsname{\def\PY@tc##1{\textcolor[rgb]{0.00,0.00,1.00}{##1}}}
\expandafter\def\csname PY@tok@vc\endcsname{\def\PY@tc##1{\textcolor[rgb]{0.10,0.09,0.49}{##1}}}
\expandafter\def\csname PY@tok@vg\endcsname{\def\PY@tc##1{\textcolor[rgb]{0.10,0.09,0.49}{##1}}}
\expandafter\def\csname PY@tok@vi\endcsname{\def\PY@tc##1{\textcolor[rgb]{0.10,0.09,0.49}{##1}}}
\expandafter\def\csname PY@tok@vm\endcsname{\def\PY@tc##1{\textcolor[rgb]{0.10,0.09,0.49}{##1}}}
\expandafter\def\csname PY@tok@sa\endcsname{\def\PY@tc##1{\textcolor[rgb]{0.73,0.13,0.13}{##1}}}
\expandafter\def\csname PY@tok@sb\endcsname{\def\PY@tc##1{\textcolor[rgb]{0.73,0.13,0.13}{##1}}}
\expandafter\def\csname PY@tok@sc\endcsname{\def\PY@tc##1{\textcolor[rgb]{0.73,0.13,0.13}{##1}}}
\expandafter\def\csname PY@tok@dl\endcsname{\def\PY@tc##1{\textcolor[rgb]{0.73,0.13,0.13}{##1}}}
\expandafter\def\csname PY@tok@s2\endcsname{\def\PY@tc##1{\textcolor[rgb]{0.73,0.13,0.13}{##1}}}
\expandafter\def\csname PY@tok@sh\endcsname{\def\PY@tc##1{\textcolor[rgb]{0.73,0.13,0.13}{##1}}}
\expandafter\def\csname PY@tok@s1\endcsname{\def\PY@tc##1{\textcolor[rgb]{0.73,0.13,0.13}{##1}}}
\expandafter\def\csname PY@tok@mb\endcsname{\def\PY@tc##1{\textcolor[rgb]{0.40,0.40,0.40}{##1}}}
\expandafter\def\csname PY@tok@mf\endcsname{\def\PY@tc##1{\textcolor[rgb]{0.40,0.40,0.40}{##1}}}
\expandafter\def\csname PY@tok@mh\endcsname{\def\PY@tc##1{\textcolor[rgb]{0.40,0.40,0.40}{##1}}}
\expandafter\def\csname PY@tok@mi\endcsname{\def\PY@tc##1{\textcolor[rgb]{0.40,0.40,0.40}{##1}}}
\expandafter\def\csname PY@tok@il\endcsname{\def\PY@tc##1{\textcolor[rgb]{0.40,0.40,0.40}{##1}}}
\expandafter\def\csname PY@tok@mo\endcsname{\def\PY@tc##1{\textcolor[rgb]{0.40,0.40,0.40}{##1}}}
\expandafter\def\csname PY@tok@ch\endcsname{\let\PY@it=\textit\def\PY@tc##1{\textcolor[rgb]{0.25,0.50,0.50}{##1}}}
\expandafter\def\csname PY@tok@cm\endcsname{\let\PY@it=\textit\def\PY@tc##1{\textcolor[rgb]{0.25,0.50,0.50}{##1}}}
\expandafter\def\csname PY@tok@cpf\endcsname{\let\PY@it=\textit\def\PY@tc##1{\textcolor[rgb]{0.25,0.50,0.50}{##1}}}
\expandafter\def\csname PY@tok@c1\endcsname{\let\PY@it=\textit\def\PY@tc##1{\textcolor[rgb]{0.25,0.50,0.50}{##1}}}
\expandafter\def\csname PY@tok@cs\endcsname{\let\PY@it=\textit\def\PY@tc##1{\textcolor[rgb]{0.25,0.50,0.50}{##1}}}

\def\PYZbs{\char`\\}
\def\PYZus{\char`\_}
\def\PYZob{\char`\{}
\def\PYZcb{\char`\}}
\def\PYZca{\char`\^}
\def\PYZam{\char`\&}
\def\PYZlt{\char`\<}
\def\PYZgt{\char`\>}
\def\PYZsh{\char`\#}
\def\PYZpc{\char`\%}
\def\PYZdl{\char`\$}
\def\PYZhy{\char`\-}
\def\PYZsq{\char`\'}
\def\PYZdq{\char`\"}
\def\PYZti{\char`\~}
% for compatibility with earlier versions
\def\PYZat{@}
\def\PYZlb{[}
\def\PYZrb{]}
\makeatother


    % Exact colors from NB
    \definecolor{incolor}{rgb}{0.0, 0.0, 0.5}
    \definecolor{outcolor}{rgb}{0.545, 0.0, 0.0}



    
    % Prevent overflowing lines due to hard-to-break entities
    \sloppy 
    % Setup hyperref package
    \hypersetup{
      breaklinks=true,  % so long urls are correctly broken across lines
      colorlinks=true,
      urlcolor=urlcolor,
      linkcolor=linkcolor,
      citecolor=citecolor,
      }
    % Slightly bigger margins than the latex defaults
    
    \geometry{verbose,tmargin=1in,bmargin=1in,lmargin=1in,rmargin=1in}
    
    

    \begin{document}
    
    
    \maketitle
    
    

    

\begin{center}\rule{1\linewidth}{\linethickness}\end{center}
The Course of Computational Chemistry, Peking University\\
Acknowledgement to Prof. Wenjian Liu.


    \section{Members}\label{members}

Qilin He, Zhi Zi, Haoming Liu, Ta Teng,Zuoran Qiao, Xin He(Leader)\\
 An old version refers
https://github.com/Utenaq/2018QC-Project-Ab-initio-wavefunction-program
.\\

    \section{compile and usage}\label{compile-and-usage}

\subsection{compile}\label{compile}

in the main directory or under \texttt{./src}, use \texttt{make}
command; it needs g++ compiler.

\subsection{usage}\label{usage}

\begin{longtable}[]{@{}lll@{}}
\toprule
command & term &\tabularnewline
\midrule
\endhead
\texttt{main\ -h} & get help info &\tabularnewline
\texttt{main\ -d} & default test (eg. HeH+) & test file locates in
\texttt{../test}\tabularnewline
\texttt{main\ -f\ {[}file.gjf{]}} & normal calculation & a series test
file are under \texttt{../test}\tabularnewline
\bottomrule
\end{longtable}

optional test file lies in \texttt{../test},including \textbf{H.gjf},
\textbf{He.gjf}, \textbf{H2.gjf}, \textbf{HeH.gjf}, \textbf{H4.gjf},
\textbf{CH4.gjf}.\\
use as \texttt{main\ -f\ ../test/H2.gjf}.

    \section{class and type}\label{class-and-type}

refer to \href{}{class and type}

    \section{handling of molecule
integral}\label{handling-of-molecule-integral}

    \paragraph{Radial gaussian and cartesian
gaussian}\label{radial-gaussian-and-cartesian-gaussian}
\\

radial gaussian gives as:\\
\[ g_{lmn}(\alpha, \mathbf{r}) = \Big(\sqrt{\frac{2}{\pi}} \frac{(4\alpha)^{n+1/2}}{(2n-1)!!}\Big) r^{n-1}e^{-\alpha r^2} Y_{lm}(\theta, \varphi) \]\\
but here we more often use the cartesian gaussian as:\\
\[ g(A,\alpha, l,m,n) = N x_A^l y_A^m z_A^n e^{-\alpha r_A^2} \] where
by formula\\
\[\int_{-\infty}^{\infty}x^{2n}e^{-\alpha x^2}=\sqrt{\frac{\pi}{\alpha}}\frac{(2n-1)!!}{(2\alpha)^n}\]
we have
\[ N =\big[\big(\frac{2\alpha}{\pi}\big)^{3/2}\frac{(4\alpha)^{l+m+n}}{(2l-1)!!(2m-1)!!(2n-1)!!} \big]^{1/2}\]

    \paragraph{GTO product theorem}\label{gto-product-theorem}

\[ e^{-\alpha_1 r_A^2}e^{-\alpha_2 r_B^2} = \exp\big[-\frac{\alpha_1\alpha_2}{\alpha_1+\alpha_2}\big]\exp\big[-(\alpha_1+\alpha_2)r_P^2 \big] \]
where

\[ \mathbf{P} \overset{def}{==} \frac{\alpha_1 \mathbf{A} + \alpha_2 \mathbf{B}}{\alpha_1+\alpha_2}\]
so let \(x_A^{l_1}x_B^{l_2}\) expand at \(x_P\) as:
\begin{align*}
x_A^{l_1}x_B^{l_2} &= (x_P - \overline{PA}_x)_A^{l_1} (x_P - \overline{PB}_x)_B^{l_2} 
\\
&= \sum_{i_1=0}^{l_1} \sum_{i_2=0}^{l_2} (-1)^{i_1+i_2} C_{l_1}^{i_1}C_{l_2}^{i_2} (\overline{PA}_x)^{l_1-i_1} (\overline{PB}_x)^{l_2-i_2} x_P^{i_1+i_2} 
\\
&= \sum_l^{l_1+l_2} (-1)^{l} f_l (l_1,l_2,\overline{PA}_x, \overline{PB}_x) x_P^l
\end{align*}
where
\[ 
f_l(l_1,l_2,a,b) = \sum_{i_1=0}^{l_1} \sum_{i_2=0}^{l_2} \delta_{l,l_1+l_2} C_{l_1}^{i_1}C_{l_2}^{i_2} a^{l_1-i_1} b^{l_2-i_2} 
\]
and so on:\\
\begin{align*}
&g(A,\alpha_1,l_1,m_1,n_1)g(B,\alpha_2,l_2,m_2,n_2) 
\\&= \sum_{l=0}^{l_1+l_2}\sum_{m=0}^{m_1+m_2}\sum_{n=0}^{n_1+n_2}
(-1)^{l+m+n} f_l (l_1,l_2,\overline{PA}_x, \overline{PB}_x)
f_m (m_1,m_2,\overline{PA}_y, \overline{PB}_y)
\\
&\times
f_n (n_1,n_2,\overline{PA}_z, \overline{PB}_z)
x_P^l y_P^m z_P^n 
\exp\big[-\frac{\alpha_1\alpha_2}{\alpha_1+\alpha_2}\big]\exp\big[-(\alpha_1+\alpha_2)r_P^2 \big]
\end{align*}

    \paragraph{useful integral formula}\label{useful-integral-formula}

\begin{enumerate}
\def\labelenumi{\arabic{enumi}.}
\tightlist
\item
  \[ \int_{-\infty}^{\infty} x^{2n}e^{-\alpha x^2}dx = \big(\frac{\pi}{\alpha}\big)^{1/2}
  \frac{(2n-1)!!}{(2\alpha)^n}
  \]
\item
  \[
  \int_{-\infty}^{\infty} x^{2n+1}e^{-\alpha x^2}dx = 0
  \]
\item
  \[
  \int_{-\infty}^{\infty} e^{ixy}x^{n}e^{-\alpha x^2}dx = i^n \big(\frac{\pi}{\alpha}\big)^{1/2}\big(\frac{1}{2\sqrt{\alpha}}\big)^n H_n\big(\frac{y}{2\sqrt{\alpha}}\big) e^{-y^2/4\alpha}
  \] where \(H_n\) is Hermite polynomial:\\
  \[
  H_n(z) = (-1)^n e^{z^2} \frac{d^n}{dz^n} e^{-z^2}
  = n!\sum_{i=0}^{[n/2]}\frac{(-1)^i}{i!(n-2i)!}(2z)^{n-2i}
  \]
\item
  \[
  \frac{1}{r} = \frac{1}{2\pi^2} \int e^{i\mathbf{k}\cdot\mathbf{r}}\frac{1}{k^2} d\mathbf{k}
  \]
\item
  \[
  e^{-\sigma k^2} = 2\sigma k^2 \int_0^1 S^{-3} e^{-\sigma k^2/S^2} dS
  \]
\item
  defines \[
  F_n (t) = \int_0^1 u^{2n} e^{-t u^2} du
  \]
\end{enumerate}

    \paragraph{Radial integral formula (1s
type)}\label{radial-integral-formula-1s-type}

for n=1, l=m=0, or called 1s function, such integral is derived by
\textbf{Boys} and \textbf{Shavitt}.\\
1. S integral\\
\[ S = \int e^{-a(\mathbf{r}-\mathbf{A})^2 - b(\mathbf{r}-\mathbf{B})^2} 
= \big( \frac{\pi}{a+b} \big)^{3/2} e^{-ab/(a+b)\cdot \overline{AB}^2 }
\] 2. T integral\\
\[ T = \int e^{-a(\mathbf{r}-\mathbf{A})^2}(-)\frac{1}{2}\nabla^2 e^{- b(\mathbf{r}-\mathbf{B})^2} 
= (\frac{ab}{a+b})(3-\frac{2ab}{a+b}) e^{-ab/(a+b)\cdot \overline{AB}^2 }
\] 3. V integral\\
\[ V = \int e^{-a(\mathbf{r}-\mathbf{A})^2} \frac{1}{|\mathbf{r}-\mathbf{C}|} e^{- b(\mathbf{r}-\mathbf{B})^2} 
= (\frac{2\pi}{a+b}) F_0[(a+b)\overline{CP}^2] e^{-ab/(a+b)\cdot \overline{AB}^2 }
\]\\
4. ERI integral
\begin{align*}
 I &= \int e^{-a(\mathbf{r_1}-\mathbf{A})^2 - b(\mathbf{r_1}-\mathbf{B})^2} \frac{1}{r_{12}} e^{- c(\mathbf{r_2}-\mathbf{C})^2 - c(\mathbf{r_2}-\mathbf{D})^2} 
\\
&= \frac{2\pi^{5/2}}{(a+b)(c+d)\sqrt{a+b+c+d}} F_0[\frac{(a+b)(c+d)}{a+b+c+d}\cdot\overline{PQ}^2] 
e^{-ab/(a+b)\cdot \overline{AB}^2 - cd/(c+d)\cdot \overline{CD}^2 }
\end{align*}

    \paragraph{Cartesian integral formula}\label{cartesian-integral-formula}

\begin{itemize}
\tightlist
\item
  overlap integral (S integral)\\
  by GTO product theorem, we can easily get:\\
  \[
  S =\big(\frac{\pi}{a+b}\big)^{3/2} e^{-ab/(a+b)\cdot \overline{AB}^2}\tilde{S}_{l_1l_2}
  \tilde{S}_{m_1m_2} \tilde{S}_{n_1n_2}
  \] where\\
  \[
  \tilde{S}_{l_1l_2} = \sum_{i=0}^{[\frac{l_1+l_2}{2}]} f_{2i}(l_1,l_2,\overline{PA}_x,\overline{PB}_x) \frac{(2i-1)!!}{2^i (a+b)^i}
  \]
\end{itemize}

    \begin{itemize}
\tightlist
\item
  kinetic integral (T integral)\\
  note \(g(A,a,l,m,n)\) as \(\mid aAlmn \rangle\), it's easy to show
  that its derivatives have following properties:\\
  \[\frac{\partial}{\partial x_A} \mid aAlmn \rangle = (lx_A^{-1}-2ax_A) \mid aAlmn \rangle\]
  \[\frac{\partial^2}{\partial x_A^2} \mid aAlmn \rangle = (l(l-1)x_A^{-2}-2a(2l+1)+4a^2x_A^2) \mid aAlmn \rangle \]\\
  so that: 
  \begin{align*}
  T &=
  \alpha_2 (2(l_2+m_2+n_2)+3)S_{l_1m_1n_1;l_2m_2n_2}
  \\ &-
  2\alpha_2^2 \big(S_{l_1m_1n_1;(l_2+2)m_2n_2} +
  S_{l_1m_1n_1;l_2(m_2+2)n_2} +
  S_{l_1m_1n_1;l_2m_2(n_2+2)}\big) 
  \\ &-
  \frac{1}{2}\big(l_2(l_2-1)S_{l_1m_1n_1;(l_2-2)m_2n_2} +
  m_2(m_2-1)S_{l_1m_1n_1;l_2(m_2-2)n_2} 
  \\ &+
  n_2(n_2-1)S_{l_1m_1n_1;l_2m_2(n_2-2)}\big)
  \end{align*}
\end{itemize}

    \begin{itemize}
\tightlist
\item
  nuclear Coulomb integral (V integral; without nucleus-nucleus Coulomb
  integral!)\\
  \begin{align*}
  V_{ABC} &= \int g(A,a,l_1,m_1,n_1)\frac{1}{r_C}g(B,b,l_2,m_2,n_2)d\tau
  \\
  &=N_AN_B \frac{1}{2\pi}e^{-ab/(a+b)\cdot \overline{AB}^2} \sum_{lmn} f_l(l_1,l_2,\overline{PA}_x, \overline{PB}_x) f_m(m_1,m_2,\overline{PA}_y, \overline{PB}_y) f_n(n_1,n_2,\overline{PA}_z, \overline{PB}_z)
  \\
  &\times \int e^{i\mathbf{k}\cdot \mathbf{r}_{cp}} \frac{1}{k^2}d\mathbf{k} \int e^{ik_x x}x^l e^{-(a+b)x^2} dx \int e^{ik_y y}y^m e^{-(a+b)y^2} dy \int e^{ik_y y}y^n e^{-(a+b)z^2} dz
  \end{align*}
  \\
  we here omit the derivation, giving result directly as:\\
  \begin{align*}
  V_{ABC} &= 
  \frac{2\pi}{a+b}e^{-ab/(a+b)\cdot\overline{AB}^2} \sum_{\nu}^{N}\sum_{I=0}^{l_1+l_2}
  \sum_{J=0}^{m_1+m_2}\sum_{K=0}^{n_1+n_2} 
  G_{l_1l_2}^{I}(A_x,B_x,C_x)
  \\
  &\times G_{m_1m_2}^{J}(A_y,B_y,C_y) G_{n_1n_2}^{K}(A_z,B_z,C_z)
  F_{\nu}[(a+b)\overline{PC}^2]\delta_{\nu,I+J+K}
   \end{align*} 
  where 
  \begin{align*}
  G_{l_1l_2}^{I}(A_x,B_x,C_x) = \sum_{i=0}^{l_1+l_2}\sum_{r=0}^{[i/2]} \sum_{u=0}^{[i/2]-r} (-1)^i f_i (l_1,l_2,\overline{PA}_x,\overline{PB}_x)
  \\ \times
  \frac{(-1)^u i!(\overline{PC}_x)^{i-2r-2u}}{r!u!(i-2r-2u)!} \big(\frac{1}{4(a+b)}\big)^{r+u} \delta_{I,i-2r-u}
   \end{align*}
\end{itemize}

    \begin{itemize}
\tightlist
\item
  ERI integral\\
  \begin{align*}
  I &= \frac{2\pi^2}{(a+b)(c+d)}\big(\frac{1}{a+b+c+d}\big)^{1/2} e^{-ac/(a+b)\cdot \overline{AB}^2 - cd/(c+d)\cdot \overline{CD}^2} 
  \\ 
  &\times
  \sum_{\nu=0}^{N} F_{\nu}(\overline{PQ}^2/(4\gamma))\sum_{I=0}^{l_1+l_2+l_3+l_4}
  \sum_{J=0}^{m_1+m_2+m_3+m_4}\sum_{K=0}^{n_1+n_2+n_3+n_4}
  D_{l_1l_2l_3l_4}^I (A_x,B_x,C_x,D_x)
  \\ 
  &\times
  D_{m_1m_2m_3m_4}^J (A_y,B_y,C_y,D_y)
  D_{n_1n_2n_3n_4}^K (A_z,B_z,C_z,D_z)\delta_{\nu,I+J+K}
   \end{align*}
   where
  \begin{align*}
  \gamma &= (a+b+c+d)/[4(a+b)(c+d)]
  \\
  D_{l_1l_2l_3l_4}^I (A_x,B_x,C_x,D_x) &= \sum_{l=0}^{l_1+l_2}\sum_{l'=0}^{l_3+l_4}
  \sum_{u=0}^{[(l+l')/2]} \frac{(-1)^u (l+l')! \overline{PQ}_x^{l+l'-2u}}{u!(l+l'-2u)!\gamma^{l+l'-u}}
  \\ 
  &\times
  H_{l_1l_2}^l (\overline{PA}_x,\overline{PB}_x,a+b) (-1)^{l'} 
  H_{l_3l_4}^{l'} (\overline{QC}_x,\overline{QD}_x,c+d)\delta_{I,l+l'-u}
  \end{align*}
  where 
  \[
  H_{l_1l_2}^l (\overline{PA}_x,\overline{PB}_x,\beta)
  = \sum_{i=0}^{l_1+l_2}\sum_{r=0}^{[i/2]} \frac{i!}{r!(i-2r)!(4\beta)^{i-r}} f_i (l_1,l_2,\overline{PA}_x,\overline{PB}_x)\delta_{l,i-2r}
  \]
\end{itemize}

    \section{SCF procedure}\label{scf-procedure}

    \paragraph{procedure}\label{procedure}

\begin{enumerate}
\def\labelenumi{\arabic{enumi}.}
\tightlist
\item
  read and construct basis set space.\\
\item
  calculate S, H, ERI inetgral matrix. Symmetric orthogonalize S to get
  transfrom matrix X and its inverse Y.
\item
  assume H as Fock matrix, solve the coefficents, then get initial
  density matrix P for guess.
\item
  from P, H, G matrix to claculate Fock matrix, by transform X to F'.
\item
  solve the eigenvalue problem of F', get its eigenvalues e and
  eigenvectors C', and by transform X to C.
\item
  determine the population of MOs, then calculate energy E, density
  matrix P.\\
\item
  if E and P is consistent with the last step, jump loop.
\end{enumerate}

    \section{Symmetry(lack)}\label{symmetrylack}

    \section{Test and result}\label{test-and-result}

\begin{longtable}[]{@{}llll@{}}
\toprule
test & result {[}au{]} & Gauss09 {[}au{]} & remark\tabularnewline
\midrule
\endhead
H & -0.42244193 & -0.4982329 & not suit for open-shell\tabularnewline
He & -2.85516043 & -2.8551604 & yes\tabularnewline
H2 & -1.11003090 & -1.1100309 & yes\tabularnewline
HeH+ & -2.89478689 & -2.8947869 & yes\tabularnewline
H4 & -1.80246920 & -1.7251712 & no\tabularnewline
CH4 & vibration & -39.9119255 & no\tabularnewline
\bottomrule
\end{longtable}


    % Add a bibliography block to the postdoc
    
    
    
    \end{document}

